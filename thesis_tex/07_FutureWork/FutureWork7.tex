\chapter{Future Work}
\label{chap:futurework}


So far, the pose estimation system deals with an isolated object. In the future, in order to test the pose estimation system in bin-picking scenarios-careful attention should be given to the steps of features descriptors and matching strategy. These two are fundamental in dealing with occlusion and scene with clutter which are common in bin-picking scenarios. The proposed methods were evaluated with two cameras. Unfortunately, a third camera (RealSense D415) which is supposed to be more accurate, came to the laboratory a few days before submitting this thesis work. However, the author of this thesis managed to carry on a third experiment, where a small database of source cloud was generated, and the registration method was evaluated by calculating two metrics: fitness, which measures the overlapping area, and the Root Mean Square Error, which measures the standard deviation of the residuals.
 The results were satisfactory and acceptable for this type of task, while performing bettere than the Astra camera during the whole process of validation testing. The validation system results can be seen in  \ref{appendix...???????????...}. However, a more thorough evaluation should be given to this sensor for future applications.  


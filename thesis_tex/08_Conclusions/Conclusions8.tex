\chapter{Conclusions}
\label{chap:con}


The thesis has proposed a practical and simple method for robot-camera calibration as well as a system for localizing an aisolated part. The robot-camera calibration also refered to as an Eye-To-Hand calibration is based on the robot moving a standard calibration checkerboard. The method was implemented in ROS using the tools and libraries available which made it easy the compute the pose of camera relative to the robot frame given the fact that a tf tree of the system is known in advance. \\
When addressing the issue of localizing a 3D object by using visual information to control a robotic arm, the most accurate parameters of the camera calibration (Intrinsic and eye-to-hand calibration)  were selected by means of reprojection errors, for the case of intrisinc parameters. And repeatability test for the case of the camera pose relative to the robot frame. \\
After those, the 3D object pose estimation system has been implemented and evaluated for a single isolated industrial object in depth image taken from the sensory device used. The system uses point cloud data generated from a 3D CAD model and describes its characteristics by using local feature descriptors. These are then matched with the descriptors of the point cloud data generated from the scene (depth image) in order to find the 6-DoF pose of the model in the robot coordinate system. This initial pose estimation is then refined by a registration method, such as ICP.\\
For the validation of the system, two experiments were executed. The Astra and RealSense cameras were used for the purpose of comparison. In the first experiment,  the source cloud is generated from the 3D CAD model, while in the second experiment, the source cloud, also known as a reference, is generated from the scene with the camera used.\\ 
In both experiments, different displacement and angles were applied to the industrial object which was located on a well defined coordinate system given by the checkerboard. \\
The pose of the checkerboard relative to the robot frame was determined with the use of the robot's TCP. By knowing the displacements and angles applied to the object, the ground truth was estimated by visual inspection.  After analyzing the outcomes of both experiments, it was concluded that the RealSense camera is not suitable for the task of pose estimation where a high quality of point cloud is needed. As to the Astra camera, it was proven to show good performance when the reference cloud is generated by the camera itself. \\


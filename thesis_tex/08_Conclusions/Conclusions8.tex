\chapter{underconstruction Conclusions}
\label{chap:con}


The thesis has proposed a practical and flexible method for robot-camera calibration as well as a system for localizing an aisolated part. In addition to that, an intuitive way to determine ground truth was showed.\\

This thesis has developed and validated a simple and flexible robot-camera calibration method. It is based on the robot moving a standard calibration checkerboard. \\



In addition to that, the thesis has implemented and evaluated a system for a 3D object pose estimation of a single isolated industrial object in depth image taken from a camera. The system uses point cloud data generated from a 3D CAD model and describes its characteristics by using local feature descriptors. These are then matched with the descriptors of the point cloud data generated from the scene in order to find the 6-DoF pose of the model in the robot coordinate system. This initial pose estimation is then refined by a registration method, such as ICP.  For the validation of the system, two experiments were executed. The Astra and RealSense cameras were used for the purpose of comparison. In the first experiment,  the source cloud is generated from the 3D CAD model, while in the second experiment, the source cloud, also known as a reference, is generated from the scene with the camera used. In both experiments, different displacement and angles were applied to the industrial object which was located on a well defined coordinate system given by the checkerboard. The pose of the checkerboard relative to the robot frame was determined with the use of the robot's TCP. By knowing the displacements and angles applied to the object, the ground truth was estimated by visual inspection.  After analyzing the outcomes of both experiments, it was concluded that the RealSense camera is not suitable for the task of pose estimation where a high quality of point cloud is needed. As to the Astra camera, it was proven to show good performance when the reference cloud is generated by the camera itself. \\





\iffalse
A robot-camera calibration is performed also.
The contributions of this thesis are as follows: The system uses FPFH (Fast Point Feature Histogram) for describing the local region and a hypothesize-and-test paradigm, e.g. RANSAC in the matching process. In contrast to several approaches those whose rely on Point Pair Features as feature descriptors and a geometry hashing, e.g. voting-scheme as the matching process.
\fi